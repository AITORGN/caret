\documentclass[12 pt]{beamer}\usepackage[]{graphicx}\usepackage[]{color}
%% maxwidth is the original width if it is less than linewidth
%% otherwise use linewidth (to make sure the graphics do not exceed the margin)
\makeatletter
\def\maxwidth{ %
  \ifdim\Gin@nat@width>\linewidth
    \linewidth
  \else
    \Gin@nat@width
  \fi
}
\makeatother

\definecolor{fgcolor}{rgb}{0.196, 0.196, 0.196}
\newcommand{\hlnum}[1]{\textcolor[rgb]{0.063,0.58,0.627}{#1}}%
\newcommand{\hlstr}[1]{\textcolor[rgb]{0.063,0.58,0.627}{#1}}%
\newcommand{\hlcom}[1]{\textcolor[rgb]{0.588,0.588,0.588}{#1}}%
\newcommand{\hlopt}[1]{\textcolor[rgb]{0.196,0.196,0.196}{#1}}%
\newcommand{\hlstd}[1]{\textcolor[rgb]{0.196,0.196,0.196}{#1}}%
\newcommand{\hlkwa}[1]{\textcolor[rgb]{0.231,0.416,0.784}{#1}}%
\newcommand{\hlkwb}[1]{\textcolor[rgb]{0.627,0,0.314}{#1}}%
\newcommand{\hlkwc}[1]{\textcolor[rgb]{0,0.631,0.314}{#1}}%
\newcommand{\hlkwd}[1]{\textcolor[rgb]{0.78,0.227,0.412}{#1}}%

\usepackage{framed}
\makeatletter
\newenvironment{kframe}{%
 \def\at@end@of@kframe{}%
 \ifinner\ifhmode%
  \def\at@end@of@kframe{\end{minipage}}%
  \begin{minipage}{\columnwidth}%
 \fi\fi%
 \def\FrameCommand##1{\hskip\@totalleftmargin \hskip-\fboxsep
 \colorbox{shadecolor}{##1}\hskip-\fboxsep
     % There is no \\@totalrightmargin, so:
     \hskip-\linewidth \hskip-\@totalleftmargin \hskip\columnwidth}%
 \MakeFramed {\advance\hsize-\width
   \@totalleftmargin\z@ \linewidth\hsize
   \@setminipage}}%
 {\par\unskip\endMakeFramed%
 \at@end@of@kframe}
\makeatother

\definecolor{shadecolor}{rgb}{.97, .97, .97}
\definecolor{messagecolor}{rgb}{0, 0, 0}
\definecolor{warningcolor}{rgb}{1, 0, 1}
\definecolor{errorcolor}{rgb}{1, 0, 0}
\newenvironment{knitrout}{}{} % an empty environment to be redefined in TeX

\usepackage{alltt}

\usepackage{beamerthemesplit} 
\usepackage{fancyvrb}

%%%%%%%%%%%%%%%%%%%%%%%%%%%%%%%%%%%%%%%%%%%%%%%%%%%%%%%%%%%%%%%%%%%%%%%%%%%%%%%%%%%%

\mode<presentation> {
  \usetheme{Boadilla} 
  \usecolortheme{whale}
  \setbeamercovered{transparent}
}

%%%%%%%%%%%%%%%%%%%%%%%%%%%%%%%%%%%%%%%%%%%%%%%%%%%%%%%%%%%%%%%%%%%%%%%%%%%%%%%%%%%%
%% Custom definitions

\definecolor{darkgreen}{rgb}{0,0.3,0}
\definecolor{darkred}{rgb}{0.6,0.0,0}
\definecolor{darkblue}{rgb}{0,0,0.8}

\newcommand{\mxnum}[1]{\texttt{\hlnum{#1}}}%
\newcommand{\mxkwc}[1]{\texttt{\hlkwc{#1}}}%
\newcommand{\mxkwd}[1]{\texttt{\hlkwd{#1}}}%

\newcommand{\pkg}[1]{{\fontseries{b}\selectfont #1}}
\renewcommand{\pkg}[1]{{\color{darkgreen}\textsf{#1}}}

%%%%%%%%%%%%%%%%%%%%%%%%%%%%%%%%%%%%%%%%%%%%%%%%%%%%%%%%%%%%%%%%%%%%%%%%%%%%%%%%%%%%
%% 




%%%%%%%%%%%%%%%%%%%%%%%%%%%%%%%%%%%%%%%%%%%%%%%%%%%%%%%%%%%%%%%%%%%%%%%%%%%%%%%%%%%%

\title{Nobody Knows What It's Like To Be the Bad Man}
\author{Max Kuhn, Ph.D}
\institute{Pfizer Global R$\&$D \linebreak Groton, CT\linebreak max.kuhn@pfizer.com }
\date{}
 
%%%%%%%%%%%%%%%%%%%%%%%%%%%%%%%%%%%%%%%%%%%%%%%%%%%%%%%%%%%%%%%%%%%%%%%%%%%%%%%%%%%%
\IfFileExists{upquote.sty}{\usepackage{upquote}}{}
\begin{document}

\begin{frame}[plain]
  \maketitle
\end{frame}

\title{caret}
\author{Max Kuhn}
\institute{Pfizer}

%%%%%%%%%%%%%%%%%%%%%%%%%%%%%%%%%%%%%%%%%%%%%%%%%%%%%%%%%%%%%%%%%%%%%%%%%%%%%%%%%%%%

\begin{frame}
  \frametitle{Outline}

\begin{itemize}
\item What is the \pkg{caret} package?
\item What makes it different
\item Version control
\item The CRAN release process
\item Testing
\item Documentation
\end{itemize}

\end{frame}



%%%%%%%%%%%%%%%%%%%%%%%%%%%%%%%%%%%%%%%%%%%%%%%%%%%%%%%%%%%%%%%%%%%%%%
  
  \begin{frame}[fragile]
\frametitle{Model Function Consistency}

Since there are many modeling packages in R written by different people,
there are some inconsistencies in how models are specified and
predictions are created.

\vspace{.15in}

For example, many models have only one method of specifying the model
(e.g. formula method only)


\end{frame}

%%%%%%%%%%%%%%%%%%%%%%%%%%%%%%%%%%%%%%%%%%%%%%%%%%%%%%%%%%%%%%%%%%%%%%
  
  \begin{frame}[fragile]
\frametitle{Generating Class Probabilities Using Different Packages}

\begin{footnotesize}
\begin{center}
\begin{tabular}{rcl}
{\bf Function} && {\bf {\tt predict} Function Syntax} \\
\hline
 \href{http://cran.r-project.org/web/packages/MASS/index.html}{\pkg{MASS}}\texttt{:::lda} && {\tt \hlkwd{predict}\hlstd{(obj)}} (no options needed)\\
\pkg{stats}\texttt{:::glm} && {\tt \hlkwd{predict}\hlstd{(obj,} \hlkwc{type} \hlstd{=} \hlstr{"response"}\hlstd{)}} \\
\href{http://cran.r-project.org/web/packages/gbm/index.html}{\pkg{gbm}}\texttt{:::gbm} && {\tt \hlkwd{predict}\hlstd{(obj,} \hlkwc{type} \hlstd{=} \hlstr{"response"}\hlstd{, n.trees)}} \\
\href{http://cran.r-project.org/web/packages/mda/index.html}{\pkg{mda}}\texttt{:::mda} && {\tt \hlkwd{predict}\hlstd{(obj,} \hlkwc{type} \hlstd{=} \hlstr{"posterior"}\hlstd{)}} \\  
\href{http://cran.r-project.org/web/packages/rpart/index.html}{\pkg{rpart}}\texttt{:::rpart} && {\tt \hlkwd{predict}\hlstd{(obj,} \hlkwc{type} \hlstd{=} \hlstr{"prob"}\hlstd{)}}   \\
\href{http://cran.r-project.org/web/packages/RWeka/index.html}{\pkg{RWeka}}\texttt{:::Weka} && {\tt \hlkwd{predict}\hlstd{(obj,} \hlkwc{type} \hlstd{=} \hlstr{"probability"}\hlstd{)}}  \\    
\href{http://cran.r-project.org/web/packages/caTools/index.html}{\pkg{caTools}}\texttt{:::LogitBoost}  && {\tt \hlkwd{predict}\hlstd{(obj,} \hlkwc{type} \hlstd{=}  \hlstr{"raw"}\hlstd{, nIter)}} \\
\hline \\
\\
\end{tabular}
\end{center}
\end{footnotesize}
\end{frame}

%%%%%%%%%%%%%%%%%%%%%%%%%%%%%%%%%%%%%%%%%%%%%%%%%%%%%%%%%%%%%%%%%%%%%%
  
  \begin{frame}[fragile]
\frametitle{The \pkg{caret} Package}

The \href{http://cran.r-project.org/web/packages/caret/index.html}{\pkg{caret}}  package was developed to:
  \begin{itemize}
\item create a unified interface for modeling and prediction
(interfaces to 180 models)
\item streamline model tuning using resampling
\item provide a variety of ``helper'' functions and classes for day--to--day model building tasks 
\item increase computational efficiency using parallel processing  
\end{itemize}

\vspace{.07in}

First commits within Pfizer: 6/2005, First version on CRAN: 10/2007

\vspace{.06in}

Website: \href{http://topepo.github.io/caret/}{http://topepo.github.io/caret/}

\vspace{.06in}

JSS Paper: \href{http://www.jstatsoft.org/v28/i05/paper}{http://www.jstatsoft.org/v28/i05/paper}

\vspace{.06in}

Model List: \href{http://topepo.github.io/caret/bytag.html}{http://topepo.github.io/caret/bytag.html}

\vspace{.06in}

Many computing sections in APM

\end{frame}

%%%%%%%%%%%%%%%%%%%%%%%%%%%%%%%%%%%%%%%%%%%%%%%%%%%%%%%%%%%%%%%%%%%%%%
  
  \begin{frame}[fragile]
\frametitle{Package Dependencies}



One thing that makes \pkg{caret} different from most other packages is that is uses code from an abnormally large number ($> 80$) of other packages. 

\vspace{.15in}

Briefly, these were in the \texttt{Depends} field of the \textt{DESCRIPTION} file which cause all of them to be loaded with \pkg{caret}.

\vspace{.15in}

For many years, they were moved to \texttt{Suggests}, which solved that issue, 


\vspace{.15in}

However, their formal dependency in the \textt{DESCRIPTION} file required CRAN  to install hundreds of other packages to check  \pkg{caret}. They were not pleased. 

\end{frame}


%%%%%%%%%%%%%%%%%%%%%%%%%%%%%%%%%%%%%%%%%%%%%%%%%%%%%%%%%%%%%%%%%%%%%%
  
  \begin{frame}[fragile]
\frametitle{The Basic Release Process}

\begin{enumerate}
  \item create a few dynamic man pages
  \item use {\color{darkred} \texttt{R CMD check --as-cran}} to ensure passing CRAN tests and {\color{darkred} unit tests}
  \item update all packages (and R)
  \item run {\color{darkred} regression tests} and evaluate results
  \item send to CRAN
  \item repeat
  \item repeat
  \item install passed \pkg{caret} version 
  \item generate {\color{darkred} HTML documentation} and sync github io branch
  \item profit!
\end{enumerate}

\end{frame}

%%%%%%%%%%%%%%%%%%%%%%%%%%%%%%%%%%%%%%%%%%%%%%%%%%%%%%%%%%%%%%%%%%%%%%
  
  \begin{frame}[fragile]
\frametitle{Required ``Optimizations''}


For example, there is one check that produces a large number of false positive warnings. For example:

\begin{knitrout}\scriptsize
\definecolor{shadecolor}{rgb}{1, 1, 1}\color{fgcolor}\begin{kframe}
\begin{alltt}
\hlstd{> }\hlstd{bwplot.diff.resamples} \hlkwb{<-} \hlkwa{function} \hlstd{(}\hlkwc{x}\hlstd{,} \hlkwc{data}\hlstd{,} \hlkwc{metric} \hlstd{= x}\hlopt{$}\hlstd{metric,} \hlkwc{...}\hlstd{)  \{}
\hlstd{+ }    \hlcom{## some code}
\hlstd{+ }    \hlstd{plotData} \hlkwb{<-} \hlkwd{subset}\hlstd{(plotData, Metric} \hlopt \hlstd{metric)}
\hlstd{+ }    \hlcom{## more code}
\hlstd{+ }\hlstd{\}}
\end{alltt}
\end{kframe}
\end{knitrout}

will trigger a wanring that ``{\tt \small bwplot.diff.resamples: no visible binding for global variable 'Metric'}''.

\vspace{.15in}

The ``solution'' is to have a file that is sourced first in the package (e.g. \texttt{aaa.R}) with the line

\begin{knitrout}\scriptsize
\definecolor{shadecolor}{rgb}{1, 1, 1}\color{fgcolor}\begin{kframe}
\begin{alltt}
\hlstd{> }\hlstd{Metric} \hlkwb{<-} \hlkwa{NULL}
\end{alltt}
\end{kframe}
\end{knitrout}

\end{frame}



%%%%%%%%%%%%%%%%%%%%%%%%%%%%%%%%%%%%%%%%%%%%%%%%%%%%%%%%%%%%%%%%%%%%%%
  
\begin{frame}[fragile]
\frametitle{Severity of Problems}

It's hard to tell which warnings should be ignored and which should not. There is also the issue of inconsistencies related to who is ``on duty'' when you submit your package. 

\vspace{.1in}

For example, I recently updated the \pkg{desirability} package and received this warning:

\vspace{.1in}



\end{frame}




%%%%%%%%%%%%%%%%%%%%%%%%%%%%%%%%%%%%%%%%%%%%%%%%%%%%%%%%%%%%%%%%%%%%%%
  
\begin{frame}[fragile]
\frametitle{Package Dependencies}

This problem was somewhat alleviated at the end of 2013 when {\em custom methods} were introduced into the package. 

\vspace{.1in}

Although this functionality had already existed in the package for some time, it was refactored to be more user freindly. 

\vspace{.1in}

In the process, much of the modeling code was moved out of \pkg{caret}'s R files and into R objects, eliminating the formal dependencies. 

\vspace{.1in}

Right now, the {\em total} number of dependencies is much smaller (2 \texttt{Depends}, 7 \texttt{Imports}, and 25 \texttt{Suggests}). 

\vspace{.1in}

This still affects testing though (described later). Also:

\vspace{.1in}

{\tt \color{darkblue} \footnotesize 1 package is needed for this model and is not installed. (gbm). Would you like to try to install it now?}

\end{frame}


%%%%%%%%%%%%%%%%%%%%%%%%%%%%%%%%%%%%%%%%%%%%%%%%%%%%%%%%%%%%%%%%%%%%%%
  
  \begin{frame}[fragile]
\frametitle{Regression Testing}

Prior to CRAN release (or whenever required), a comprehensive set of regression tests are conducted.

\vspace{.1in}

All modeling packages are updated to their current CRAN versions.

\vspace{.1in}

For each model accessed by \mxkwd{train}, \mxkwd{rfe}, and/or \mxkwd{sbf}, a set of test cases are computed with the production version of \pkg{caret} and the devle version. 

\vspace{.1in}

First, test cases are evaluated to make sure that nothing has been broken by updated versions of the consistuent packages. 

\vspace{.1in}

Diffs of the model results are computed to assess any differences in \pkg{caret} versions.

\vspace{.1in}

This process takes approximately 3hrs to complete using \texttt{make -j 12} on a Mac Pro. 

\end{frame}


%%%%%%%%%%%%%%%%%%%%%%%%%%%%%%%%%%%%%%%%%%%%%%%%%%%%%%%%%%%%%%%%%%%%%%
  
  \begin{frame}[fragile]
\frametitle{Regression Testing}

\begin{Verbatim}[fontsize=\footnotesize]
$ R CMD BATCH make_model_Rd.R
$ cd ~/tmp/2015_04_19_09__6.0-41/
$ make -j 7 -i 
 2015-04-19 09:13:44: Starting ada
 2015-04-19 09:13:44: Starting AdaBag
 2015-04-19 09:13:44: Starting AdaBoost.M1
 2015-04-19 09:13:44: Finished ada
 2015-04-19 09:13:44: Starting ANFIS
                :
make: [FH.GBML.RData] Error 1 (ignored)                
                :
 2015-04-19 12:03:52: Finished WM
 2015-04-19 12:04:48: Finished xyf
\end{Verbatim}

\end{frame}



%%%%%%%%%%%%%%%%%%%%%%%%%%%%%%%%%%%%%%%%%%%%%%%%%%%%%%%%%%%%%%%%%%%%%%
  
  \begin{frame}[fragile]
\frametitle{Documentation}

\pkg{caret} originally contained four package vignettes with in--depth descriptions of functionality with examples. 

\vspace{.15in}

Although this functionality had already existed in the package for some time, it was refactored to be more user freindly. 

\vspace{.15in}

However, this added time to \texttt{R CMD check} and was a general pain for CRAN.

\vspace{.15in}

Efforts to make the vingettes more computationally efficient (e.g. reducing the number of examples, resamples, etc.) diminished the effectiveness of the documentation.  


\end{frame}

%%%%%%%%%%%%%%%%%%%%%%%%%%%%%%%%%%%%%%%%%%%%%%%%%%%%%%%%%%%%%%%%%%%%%%
  
  \begin{frame}[fragile]
\frametitle{Documentation}

The documentation was moved out of the package and to the github IO page. 

\vspace{.15in}


These pages are built using \pkg{knitr} whenever a new version is sent to CRAN. Some advantages are:

\begin{itemize}
\item longer and more relavant examples are availible
\item update scheduile is under my control
\item dynamic documentation (e.g. D3 network graphs, JS tables)
\item better formatting
\end{itemize}

\vspace{.1in}

It currently takes about 4hr to create these (using parallel processing when possible). 

\end{frame}







\end{document}













